\documentclass[11pt, a4paper,parskip=half]{scrartcl}
\usepackage[utf8]{inputenc}
\usepackage{geometry}
\usepackage{graphics}
\usepackage{subcaption}
\usepackage{eurosym}
\usepackage[ngerman]{babel}
\usepackage{makeidx}

\usepackage{tikz}
\usetikzlibrary{shapes.geometric}

\usepackage{hyperref}
\hypersetup{colorlinks=true, linkcolor=blue, filecolor=blue, urlcolor=blue}
\urlstyle{same}


\title{DMR -- Eine Einführung}
\author{Hannes Matuschek, DM3MAT,\\\texttt{dm3mat [at] darc [dot] de}}
\date{\today}

\newcommand{\repeater}[3]{%
 \node ({#1}) at ({#2}) {%
  \begin{tikzpicture}%
   \draw [black,thick] (-.25,0) -- (0,0.5) -- (0.25,0) -- (-0.25,0);%
   \draw [black,thick,domain=-45:225] plot ({0.2*cos(\x)}, {0.5+0.2*sin(\x)});%
   \draw [black,thick,domain=-45:225] plot ({0.4*cos(\x)}, {0.5+0.4*sin(\x)});%
   \node (xxx) at (0,-.2) {{#3}};%
  \end{tikzpicture}%
 } %
}

\newcommand{\activerepeater}[3]{%
 \node ({#1}) at ({#2}) {%
  \begin{tikzpicture}%
   \draw [black,thick] (-.25,0) -- (0,0.5) -- (0.25,0) -- (-0.25,0);%
   \draw [red,thick,domain=-45:225] plot ({0.2*cos(\x)}, {0.5+0.2*sin(\x)});%
   \draw [red,thick,domain=-45:225] plot ({0.4*cos(\x)}, {0.5+0.4*sin(\x)});%
   \node (xxx) at (0,-.2) {{#3}};%
  \end{tikzpicture}%
 } %
}


\newcommand{\user}[3]{%
 \node ({#1}) at ({#2}) {%
  \begin{tikzpicture}%
   \draw [black,fill=black] (-.25,0) -- (0,0.5) -- (0.25,0) -- (-0.25,0);%
   \draw [black,fill=black] (0,.5) circle (.2); %
   \node (xxx) [text width=0.6cm, align=center] at (-.35cm,-.4) {{#3}};%
  \end{tikzpicture}%
 } %
}

\newcommand{\activeuser}[3]{%
 \node ({#1}) at ({#2}) {%
  \begin{tikzpicture}%
   \draw [red,fill=red] (-.25,0) -- (0,0.5) -- (0.25,0) -- (-0.25,0);%
   \draw [red,fill=red] (0,.5) circle (.2); %
   \node (xxx) [text width=0.6cm, align=center] at (-.35cm,-.4) {{#3}};%
  \end{tikzpicture}%
 } %
}

\newenvironment{merke}{\begin{quotation}}{\end{quotation}}   

\newcommand{\adef}[1]{\emph{#1}\index{#1}}
\newcommand{\aref}[1]{#1\index{#1}}

\makeindex

\begin{document}

\begin{titlepage}
 \maketitle
 \begin{abstract}
  Dieses Script soll eine Einführung in DMR (digital mobile radio) für den 
  unbedarften Funkamateur oder jeden Interessierten sein. Ich versuche dem 
  Leser Details solange zu verheimlichen, bis es absolut notwendig wird 
  diese zu erklären. Die meisten
  Einführungen in DMR, die ich bisher gesehen habe, sind eher eine lange Liste
  von Begriffserklärungen, die ohne Erfahrung mit DMR schwer zu verstehen sind. 
  
  Viel der empfundenen Komplexität von DMR, rührt aus dem Ursprung dieser 
  Technik. DMR wurde für den kommerziellen Funk auf Großveranstaltungen oder 
  großen industriellen Anlagen entwickelt, auch 
  \href{https://de.wikipedia.org/wiki/B\%C3\%BCndelfunk}{Bündelfunk} genannt. 
  Ich werde daher damit beginnen wofür DMR entwickelt wurde und fange erst dann
  an zu erklären wie DMR für den Amateurfunk eingesetzt wird.
 \end{abstract}
 \vfill
 \tableofcontents
\end{titlepage}

\section{Vorwissen: Relaisbetrieb} \label{sec:vorwissen} \index{Relaisbetrieb}
In diesem Abschnitt werde ich kurz den \emph{klassischen} FM-Relaisbetrieb auf VHF und UHF im Amateurfunk beschreiben. Die allermeisten lizenzierten Funkamateure werde dies noch aus der Prüfung zur Betriebstechnik oder aus eigener Erfahrung wissen. 

Wenn Sie sich aber für Amateurfunk interessieren oder selbst noch keine Erfahrung mit dem Relaisbetrieb haben, empfehle ich Ihnen diesen Abschnitt zu lesen. 

\begin{figure}[!ht]
 \centering
 \documentclass{standalone}
\usepackage{tikz}
\usetikzlibrary{shapes.geometric}
\newcommand{\repeater}[3]{%
 \node ({#1}) at ({#2}) {%
  \begin{tikzpicture}%
   \draw [black,thick] (-.25,0) -- (0,0.5) -- (0.25,0) -- (-0.25,0);%
   \draw [black,thick,domain=-45:225] plot ({0.2*cos(\x)}, {0.5+0.2*sin(\x)});%
   \draw [black,thick,domain=-45:225] plot ({0.4*cos(\x)}, {0.5+0.4*sin(\x)});%
   \node (xxx) at (0,-.2) {{#3}};%
  \end{tikzpicture}%
 } %
}

\newcommand{\activerepeater}[3]{%
 \node ({#1}) at ({#2}) {%
  \begin{tikzpicture}%
   \draw [black,thick] (-.25,0) -- (0,0.5) -- (0.25,0) -- (-0.25,0);%
   \draw [red,thick,domain=-45:225] plot ({0.2*cos(\x)}, {0.5+0.2*sin(\x)});%
   \draw [red,thick,domain=-45:225] plot ({0.4*cos(\x)}, {0.5+0.4*sin(\x)});%
   \node (xxx) at (0,-.2) {{#3}};%
  \end{tikzpicture}%
 } %
}


\newcommand{\user}[3]{%
 \node ({#1}) at ({#2}) {%
  \begin{tikzpicture}%
   \draw [black,fill=black] (-.25,0) -- (0,0.5) -- (0.25,0) -- (-0.25,0);%
   \draw [black,fill=black] (0,.5) circle (.2); %
   \node (xxx) [text width=0.6cm, align=center] at (-.35cm,-.4) {{#3}};%
  \end{tikzpicture}%
 } %
}

\newcommand{\activeuser}[3]{%
 \node ({#1}) at ({#2}) {%
  \begin{tikzpicture}%
   \draw [red,fill=red] (-.25,0) -- (0,0.5) -- (0.25,0) -- (-0.25,0);%
   \draw [red,fill=red] (0,.5) circle (.2); %
   \node (xxx) [text width=0.6cm, align=center] at (-.35cm,-.4) {{#3}};%
  \end{tikzpicture}%
 } %
}

\begin{document}
 \begin{tikzpicture}[every node/.style={scale=.8}]
  \activeuser{U1}{0,0}{DM3MAT};
  \user{U2}{6,0}{DL2XYZ};
  \path[->] (U1) edge node[above] {$144.500 MHz$} (U2) ;
 \end{tikzpicture}
\end{document}

 \documentclass{standalone}
\usepackage{tikz}
\usetikzlibrary{shapes.geometric}
\newcommand{\repeater}[3]{%
 \node ({#1}) at ({#2}) {%
  \begin{tikzpicture}%
   \draw [black,thick] (-.25,0) -- (0,0.5) -- (0.25,0) -- (-0.25,0);%
   \draw [black,thick,domain=-45:225] plot ({0.2*cos(\x)}, {0.5+0.2*sin(\x)});%
   \draw [black,thick,domain=-45:225] plot ({0.4*cos(\x)}, {0.5+0.4*sin(\x)});%
   \node (xxx) at (0,-.2) {{#3}};%
  \end{tikzpicture}%
 } %
}

\newcommand{\activerepeater}[3]{%
 \node ({#1}) at ({#2}) {%
  \begin{tikzpicture}%
   \draw [black,thick] (-.25,0) -- (0,0.5) -- (0.25,0) -- (-0.25,0);%
   \draw [red,thick,domain=-45:225] plot ({0.2*cos(\x)}, {0.5+0.2*sin(\x)});%
   \draw [red,thick,domain=-45:225] plot ({0.4*cos(\x)}, {0.5+0.4*sin(\x)});%
   \node (xxx) at (0,-.2) {{#3}};%
  \end{tikzpicture}%
 } %
}


\newcommand{\user}[3]{%
 \node ({#1}) at ({#2}) {%
  \begin{tikzpicture}%
   \draw [black,fill=black] (-.25,0) -- (0,0.5) -- (0.25,0) -- (-0.25,0);%
   \draw [black,fill=black] (0,.5) circle (.2); %
   \node (xxx) [text width=0.6cm, align=center] at (-.35cm,-.4) {{#3}};%
  \end{tikzpicture}%
 } %
}

\newcommand{\activeuser}[3]{%
 \node ({#1}) at ({#2}) {%
  \begin{tikzpicture}%
   \draw [red,fill=red] (-.25,0) -- (0,0.5) -- (0.25,0) -- (-0.25,0);%
   \draw [red,fill=red] (0,.5) circle (.2); %
   \node (xxx) [text width=0.6cm, align=center] at (-.35cm,-.4) {{#3}};%
  \end{tikzpicture}%
 } %
}

\begin{document}
 \begin{tikzpicture}[every node/.style={scale=.8}]
  \user{U1}{0,0}{DM3MAT};
  \activeuser{U2}{6,0}{DL2XYZ};
  \path[->] (U2) edge node[above] {$144.500 MHz$} (U1) ;
 \end{tikzpicture}
\end{document}

 \caption{Einfacher Simplexbetrieb, DM3MAT sendet auf der Frequenz $144.500 MHz$ direkt zu DL2XYZ. Dieser antwortet dann auf der selben Frequenz.} \label{fig:basicsimlpex}
\end{figure}

Die meisten Verbindungen zwischen zwei Funkamateuren finden im so genannten \adef{Simplexbetrieb} statt. Das heißt, die zwei Funkamateure senden und empfangen abwechselnd auf der selben Frequenz und die Verbindung zwischen ihnen ist direkt (siehe Abb. \ref{fig:basicsimlpex}). Dies funktioniert auf Kurzwelle\footnote{Als Kurzwelle oder einfach HF (\emph{high frequency}) werden Frequenzen zwischen $3MHz$ und $30MHz$ bezeichnet.} sehr gut und man kann damit weltweite Verbindungen aufbauen. 

Auf höheren Frequenzen verhält sich die Radiowelle zunehmend wie Licht und es wird auf VHF\footnote{Als VHF (\emph{very high frequency}) werden die Frequenzen zwischen $30Mhz$ und $300MHz$ bezeichnet.} und UHF\footnote{Als UHF (\emph{ultra high frequency}) werden die Frequenzen zwischen $300Mhz$ und $3000MHz$ bezeichnet.} schwierig ohne viel Aufwand\footnote{Auch auf VHF und UHF können sehr große Entfernungen überbrückt werden, nur sind dann große Richtantennen oder ein sehr hoher Standort von Nöten.} wesentlich weiter als bis zum Horizont zu gelangen. Diese Tatsache schränkt die Reichweite gerade von Handfunkgeräten stark ein. Um dennoch einen größeren Bereich überbrücken zu können, wenn man nicht gerade über einen hohen Berg mit einer großen Antenne verfügt, können sogenannte Repeater oder Relais verwendet werden. 

\adef{Repeater}\index{Relais|seealso{Repeater}} sind automatisch arbeitende Amateurfunkstationen, die meist in exponierten Lagen (hoher Berg oder hoher Turm) installiert werden, um einen möglichst großen Bereich abdecken zu können. Ihre Aufgabe ist es, Aussendungen von Funkamateuren zu empfangen und gleichzeitig wieder auszusenden. Da diese Repeater gleichzeitig empfangen und senden müssen, können sie das nicht auf der selben Frequenz tun. Daher werden diese Repeater im sogenannten \adef{Duplexbetrieb} gefahren. Das heißt, der Repeater empfängt auf einer Frequenz (der sog. \adef{Eingabefrequenz}) und sendet eben dieses empfangende Signal gleichzeitig auf einer anderen Frequenz (der sog. \adef{Ausgabefrequenz}) wieder aus. 

\begin{figure}[!ht]
 \centering
 \documentclass{standalone}
\usepackage{tikz}
\usetikzlibrary{shapes.geometric}
\newcommand{\repeater}[3]{%
 \node ({#1}) at ({#2}) {%
  \begin{tikzpicture}%
   \draw [black,thick] (-.25,0) -- (0,0.5) -- (0.25,0) -- (-0.25,0);%
   \draw [black,thick,domain=-45:225] plot ({0.2*cos(\x)}, {0.5+0.2*sin(\x)});%
   \draw [black,thick,domain=-45:225] plot ({0.4*cos(\x)}, {0.5+0.4*sin(\x)});%
   \node (xxx) at (0,-.2) {{#3}};%
  \end{tikzpicture}%
 } %
}

\newcommand{\activerepeater}[3]{%
 \node ({#1}) at ({#2}) {%
  \begin{tikzpicture}%
   \draw [black,thick] (-.25,0) -- (0,0.5) -- (0.25,0) -- (-0.25,0);%
   \draw [red,thick,domain=-45:225] plot ({0.2*cos(\x)}, {0.5+0.2*sin(\x)});%
   \draw [red,thick,domain=-45:225] plot ({0.4*cos(\x)}, {0.5+0.4*sin(\x)});%
   \node (xxx) at (0,-.2) {{#3}};%
  \end{tikzpicture}%
 } %
}


\newcommand{\user}[3]{%
 \node ({#1}) at ({#2}) {%
  \begin{tikzpicture}%
   \draw [black,fill=black] (-.25,0) -- (0,0.5) -- (0.25,0) -- (-0.25,0);%
   \draw [black,fill=black] (0,.5) circle (.2); %
   \node (xxx) [text width=0.6cm, align=center] at (-.35cm,-.4) {{#3}};%
  \end{tikzpicture}%
 } %
}

\newcommand{\activeuser}[3]{%
 \node ({#1}) at ({#2}) {%
  \begin{tikzpicture}%
   \draw [red,fill=red] (-.25,0) -- (0,0.5) -- (0.25,0) -- (-0.25,0);%
   \draw [red,fill=red] (0,.5) circle (.2); %
   \node (xxx) [text width=0.6cm, align=center] at (-.35cm,-.4) {{#3}};%
  \end{tikzpicture}%
 } %
}

\begin{document}
 \begin{tikzpicture}[every node/.style={scale=.8}]
  \activeuser{U1}{0,0}{DM3MAT};
  \activerepeater{R1}{3,1}{DL0LDS};
  \user{U2}{6,0}{DL2XYZ};
  \path[->] (U1) edge node[above,rotate=17] {$431.9625 MHz$} (R1) ;
  \path[->] (R1) edge node[above,rotate=-17] {$439.5625 MHz$} (U2);
 \end{tikzpicture}
\end{document}

 \input{../fig/fm_duplex_b}
 \caption{Einfacher Repeaterbetrieb, DM3MAT sendet auf der Eingabefrequenz $431.9625 MHz$ zum Repeater (DB0LDS) und dieser setzt das empfangende Signal direct auf der Ausgabefrequenz $439.5625 MHz$ wieder ab. Auf dieser Frequenz kann DL2XYZ das umgesetzte Signal wieder empfangen.} \label{fig:basicrepeater}
\end{figure}

Für das konkrete Beispiel in Abbildung \ref{fig:basicrepeater} bedeutet das, dass DM3MAT auf der Repeatereingabefrequenz (hier $431.9625 MHz$) sendet. Dieses Signal wird vom Repeater (hier DB0LDS) empfangen und gleichzeitig wieder auf der Ausgabefrequenz (hier $439.5625 MHz$) ausgesandt. Diese Aussendung kann nun von DL2XYZ auf der Repeaterausgabefrequenz empfangen werden. Die Antwort von DL2XYZ an DM3MAT folgt den gleichen Weg, hier sendet DL2XYZ auf der Repeatereingabefrequenz und DM3MAT kann diese Aussendung auf der Repeaterausgabefrequenz empfangenen. Auf diese Wiese können zwei Funkamateure miteinander kommunizieren, auch wenn sie sich nicht direkt erreichen können. 

\subsection{Echolink} \label{sec:echolink} \index{Echolink}
Wenn zwei Funkamateure miteinander kommunizieren wollen, die sehr weit voneinander entfernt sind und somit nicht beide einen gemeinsamen Repeater erreichen können, gibt es die Möglichkeit zwei Repeater \emph{zusammenzuschalten}. 

\begin{figure}[!ht]
 \centering
 \documentclass{standalone}
\usepackage{tikz}
\usetikzlibrary{shapes.geometric}
\newcommand{\repeater}[3]{%
 \node ({#1}) at ({#2}) {%
  \begin{tikzpicture}%
   \draw [black,thick] (-.25,0) -- (0,0.5) -- (0.25,0) -- (-0.25,0);%
   \draw [black,thick,domain=-45:225] plot ({0.2*cos(\x)}, {0.5+0.2*sin(\x)});%
   \draw [black,thick,domain=-45:225] plot ({0.4*cos(\x)}, {0.5+0.4*sin(\x)});%
   \node (xxx) at (0,-.2) {{#3}};%
  \end{tikzpicture}%
 } %
}

\newcommand{\activerepeater}[3]{%
 \node ({#1}) at ({#2}) {%
  \begin{tikzpicture}%
   \draw [black,thick] (-.25,0) -- (0,0.5) -- (0.25,0) -- (-0.25,0);%
   \draw [red,thick,domain=-45:225] plot ({0.2*cos(\x)}, {0.5+0.2*sin(\x)});%
   \draw [red,thick,domain=-45:225] plot ({0.4*cos(\x)}, {0.5+0.4*sin(\x)});%
   \node (xxx) at (0,-.2) {{#3}};%
  \end{tikzpicture}%
 } %
}


\newcommand{\user}[3]{%
 \node ({#1}) at ({#2}) {%
  \begin{tikzpicture}%
   \draw [black,fill=black] (-.25,0) -- (0,0.5) -- (0.25,0) -- (-0.25,0);%
   \draw [black,fill=black] (0,.5) circle (.2); %
   \node (xxx) [text width=0.6cm, align=center] at (-.35cm,-.4) {{#3}};%
  \end{tikzpicture}%
 } %
}

\newcommand{\activeuser}[3]{%
 \node ({#1}) at ({#2}) {%
  \begin{tikzpicture}%
   \draw [red,fill=red] (-.25,0) -- (0,0.5) -- (0.25,0) -- (-0.25,0);%
   \draw [red,fill=red] (0,.5) circle (.2); %
   \node (xxx) [text width=0.6cm, align=center] at (-.35cm,-.4) {{#3}};%
  \end{tikzpicture}%
 } %
}

\begin{document}
 \begin{tikzpicture}[every node/.style={scale=.8}]
  \activeuser{U1}{0,0}{DM3MAT};
  \activerepeater{R1}{3,1}{DB0SP};
  \repeater{R2}{6,1}{DB0LDS};
  \user{U2}{9,0}{DL2XYZ};
  \path[->] (U1) edge node[above,rotate=17] {$DTMF: 662699$} node[below,rotate=17]{$431.825 MHz$} (R1) ;
 \end{tikzpicture}
\end{document}

 \documentclass{standalone}
\usepackage{tikz}
\usetikzlibrary{shapes.geometric}
\newcommand{\repeater}[3]{%
 \node ({#1}) at ({#2}) {%
  \begin{tikzpicture}%
   \draw [black,thick] (-.25,0) -- (0,0.5) -- (0.25,0) -- (-0.25,0);%
   \draw [black,thick,domain=-45:225] plot ({0.2*cos(\x)}, {0.5+0.2*sin(\x)});%
   \draw [black,thick,domain=-45:225] plot ({0.4*cos(\x)}, {0.5+0.4*sin(\x)});%
   \node (xxx) at (0,-.2) {{#3}};%
  \end{tikzpicture}%
 } %
}

\newcommand{\activerepeater}[3]{%
 \node ({#1}) at ({#2}) {%
  \begin{tikzpicture}%
   \draw [black,thick] (-.25,0) -- (0,0.5) -- (0.25,0) -- (-0.25,0);%
   \draw [red,thick,domain=-45:225] plot ({0.2*cos(\x)}, {0.5+0.2*sin(\x)});%
   \draw [red,thick,domain=-45:225] plot ({0.4*cos(\x)}, {0.5+0.4*sin(\x)});%
   \node (xxx) at (0,-.2) {{#3}};%
  \end{tikzpicture}%
 } %
}


\newcommand{\user}[3]{%
 \node ({#1}) at ({#2}) {%
  \begin{tikzpicture}%
   \draw [black,fill=black] (-.25,0) -- (0,0.5) -- (0.25,0) -- (-0.25,0);%
   \draw [black,fill=black] (0,.5) circle (.2); %
   \node (xxx) [text width=0.6cm, align=center] at (-.35cm,-.4) {{#3}};%
  \end{tikzpicture}%
 } %
}

\newcommand{\activeuser}[3]{%
 \node ({#1}) at ({#2}) {%
  \begin{tikzpicture}%
   \draw [red,fill=red] (-.25,0) -- (0,0.5) -- (0.25,0) -- (-0.25,0);%
   \draw [red,fill=red] (0,.5) circle (.2); %
   \node (xxx) [text width=0.6cm, align=center] at (-.35cm,-.4) {{#3}};%
  \end{tikzpicture}%
 } %
}

\begin{document}
 \begin{tikzpicture}[every node/.style={scale=.8}]
  \activeuser{U1}{0,0}{DM3MAT};
  \activerepeater{R1}{3,1}{DB0SP};
  \activerepeater{R2}{6,1}{DB0LDS};
  \user{U2}{9,0}{DL2XYZ};
  \path[->] (U1) edge node[above,rotate=17] {$431.825 MHz$} (R1) ;
  \path[->,dashed] (R1) edge node[above] {via Echolink} (R2) ;	
  \path[->] (R2) edge node[above,rotate=-17] {$439.150 MHz$} (U2) ;
 \end{tikzpicture}
\end{document}

 \documentclass{standalone}
\usepackage{tikz}
\usetikzlibrary{shapes.geometric}
\newcommand{\repeater}[3]{%
 \node ({#1}) at ({#2}) {%
  \begin{tikzpicture}%
   \draw [black,thick] (-.25,0) -- (0,0.5) -- (0.25,0) -- (-0.25,0);%
   \draw [black,thick,domain=-45:225] plot ({0.2*cos(\x)}, {0.5+0.2*sin(\x)});%
   \draw [black,thick,domain=-45:225] plot ({0.4*cos(\x)}, {0.5+0.4*sin(\x)});%
   \node (xxx) at (0,-.2) {{#3}};%
  \end{tikzpicture}%
 } %
}

\newcommand{\activerepeater}[3]{%
 \node ({#1}) at ({#2}) {%
  \begin{tikzpicture}%
   \draw [black,thick] (-.25,0) -- (0,0.5) -- (0.25,0) -- (-0.25,0);%
   \draw [red,thick,domain=-45:225] plot ({0.2*cos(\x)}, {0.5+0.2*sin(\x)});%
   \draw [red,thick,domain=-45:225] plot ({0.4*cos(\x)}, {0.5+0.4*sin(\x)});%
   \node (xxx) at (0,-.2) {{#3}};%
  \end{tikzpicture}%
 } %
}


\newcommand{\user}[3]{%
 \node ({#1}) at ({#2}) {%
  \begin{tikzpicture}%
   \draw [black,fill=black] (-.25,0) -- (0,0.5) -- (0.25,0) -- (-0.25,0);%
   \draw [black,fill=black] (0,.5) circle (.2); %
   \node (xxx) [text width=0.6cm, align=center] at (-.35cm,-.4) {{#3}};%
  \end{tikzpicture}%
 } %
}

\newcommand{\activeuser}[3]{%
 \node ({#1}) at ({#2}) {%
  \begin{tikzpicture}%
   \draw [red,fill=red] (-.25,0) -- (0,0.5) -- (0.25,0) -- (-0.25,0);%
   \draw [red,fill=red] (0,.5) circle (.2); %
   \node (xxx) [text width=0.6cm, align=center] at (-.35cm,-.4) {{#3}};%
  \end{tikzpicture}%
 } %
}

\begin{document}
 \begin{tikzpicture}[every node/.style={scale=.8}]
  \user{U1}{0,0}{DM3MAT};
  \activerepeater{R1}{3,1}{DB0SP};
  \activerepeater{R2}{6,1}{DB0LDS};
  \activeuser{U2}{9,0}{DL2XYZ};
  \path[->] (R1) edge node[above,rotate=17] {$439.425 MHz$} (U1) ;
  \path[->,dashed] (R2) edge node[above] {via Echolink} (R1) ;	
  \path[->] (U2) edge node[above,rotate=-17] {$431.875 MHz$} (R2) ;
 \end{tikzpicture}
\end{document}

 \caption{Repeaterbetrieb mit Echolink. DM3MAT verbindet die Repeater DB0SP (bei Berlin) und DB0LEI (bei Leipzig) per Echolink. Daraufhin können DM3MAT und DL2XYZ wie über einen gemeinsamen Repeater kommunizieren.} \label{fig:echolink}
\end{figure}

Diese Möglichkeit nennt sich \href{http://www.echolink.org/}{Echolink}. Dieses Netzwerk erlaubt es FM Repeater per Internet miteinander zu verbinden oder sich per Internet als einzelner Teilnehmer direkt mit einem Repeater zu verbinden. Viele FM Repeater sind in diesem Netzwerk zusammengeschlossen. 

Es ist auch häufig möglich\footnote{Dies hängt von der Konfiguration des Repeaters ab.} per Funk einen Repeater zu steuern und ihn mit einem anderen Repeater via Echolink zu verbinden. Dazu wird die sogenannte Echolink Nummer des Ziel Repeaters per DTMF Tonwahl an den Quellrepeater gesandt. Dies ist in Abbildung \ref{fig:echolink} (Oben) dargestellt. Hier sendet DM3MAT die Echolink Nummer 662699 des Relais DB0LEI bei Leipzig per DTMF an den Repeater DB0SP nahe Berlin. Dieser (DB0SP) verbindet sich dann mit dem Zielrepeater bei Leipzig (DB0LEI) über das Echolink Netzwerk. Alle weiteren Aussendungen die der Quellrepeater (DB0SP) nun empfängt werden nicht nur lokal auf der Ausgabefrequenz ausgesandt, sonder werden auch am Zielrepeater bei Leipzig (DB0LEI) ausgesandt (Abb. \ref{fig:echolink} Mitte). Somit kann DL2XYZ in Leipzig DM3MAT hören. Ebenso werden alle Aussendungen die der Zielrepeater (DB0LEI) empfängt via Echolink zum Quellrepeater bei Berlin übertragen und auch dort ausgesandt (Abb. \ref{fig:echolink} Unten).  Auf diese weise können zwei Funkamateure (in diesem Beispiel DM3MAT \& DL2XYZ), die sich nicht in der Nähe des selben Repeaters befinden, dennoch miteinander kommunizieren. 

\begin{merke}
Sobald zwei Repeater per Echolink miteinander verbunden sind, verhalten sich beide wie ein einziger Repeater. 
\end{merke}

Es gibt überall auf der Welt FM Repeater die per Echolink erreichbar sind. Dadurch ist es möglich jederzeit weltweite Kontakte mit einfachsten Mitteln (FM Handfunkgeräte mit DTMF Funktion sind ab ca. \EUR{40} erhältlich) herzustellen.

\section{DMR Einführung \& Ursprung} \label{sec:ursprung}
DMR kurz für \emph{digital mobile radio} ist ein digitaler Funkstandard für Sprech- und Datenfunk. Das heißt, die Sprache wird nicht direkt per FM  auf einem Kanal übertragen, sondern zuerst digitalisiert, mit einem verlustbehafteten Codec kodiert und erst dann als Datenpaket übertragen. Dies ermöglicht es, bei jedem Ruf\footnote{PTT Taste drücken, ins Funkgerät sprechen und dann die PTT Taste wieder loslassen.} zusätzliche Informationen wie Quelle und Ziel des Rufs mitzuübertragen.

DMR wurde als Ersatz für den analogen Bündelfunk in der kommerziellen Anwendung entwickelt. Ein klassisches Beispiel für den kommerziellen Einsatz von DMR wäre ein Verkehrsflughafen. Damit ist nicht der Flugfunk auf dem Feld und in der Luft gemeint, sondern der Funkbetrieb zwischen dem ganzen Bodenpersonal. 

Auf so einem Flughafen arbeiten sehr viele Leute mit sehr unterschiedlichen Aufgaben. Da hätten wir (ohne Anspruch auf Vollständigkeit)
\begin{itemize}
 \item Die Reinigungskolonne,
 \item die Sicherheitsleute wie Gepäckkontrolle oder Wachschutz,
 \item das Vorfeld, also die Betankung, die Gepäckverladung \& das Catering, 
 \item die Betriebsfeuerwehr und
 \item die Zentrale.
\end{itemize}

All diese Mitarbeiter bekommen ein Funkgerät und sollen die folgenden Möglichkeiten haben:
\begin{itemize}
 \item Direkte Kommunikation zur Zentrale, alle Personen sollen die Zentrale erreichen können.
 \item Direkte Kommunikation zwischen zwei Personen innerhalb ihrer Gruppe ohne das andere Gruppen gestört werden. Das heißt, die Reinigungskolonne sollte sich untereinander absprechen können, ohne die Betriebsfeuerwehr zu stören.
 \item Sogenannte Gruppenrufe einer Person an eine ganze Gruppe. Zum Beispiel ruft die Zentrale die gesamte Betriebsfeuerwehr an. Aber auch ein Anruf eines Wachschützers an alle anderen Wachschützer, um zum Beispiel Hilfe anzufordern. 
\end{itemize}
 
Gleichzeitig ist so ein Flughafen ein riesiges Gelände. Das heißt, nicht alle Mitarbeiter können alle anderen Mitarbeiter direkt erreichen. Es müssen also Repeater aufgestellt werden, damit das gesamte Gelände und alle Innenräume per Funk abgedeckt sind. Daher wird häufig in jedem Gebäude mindestens ein Repeater aufgestellt. 

Vergleicht man nun die Ansprüche dieses Kommunikationsnetzes mit dem klassischen FM-Repeaterbetrieb (Abs. \ref{sec:vorwissen}), wird schnell deutlich, dass es sehr schwierig wird dieses Konzept per analog FM-Repeater umzusetzen. Vor allem wenn mehrere Repeater in einem Netz (ähnlich Echolink) verbunden sind. Jede Kommunikation zwischen zwei Personen würde dann das gesamte Kommunikationsnetz belegen. 

Besser wäre es, wenn nur jene Repeater aktiv würden, die für die Kommunikation zwischen zwei Teilnehmern nötig sind. Dann stünden alle anderen Repeater für weitere Verbindungen bereit. Dieses Routing von Verbindungen sollte aber automatisch geschehen, da die zwei Teilnehmer nicht immer wissen werden, wo sich die jeweils andere Person befindet und somit mit welchem Repeater sie sich verbinden müssen. 

Um solche komplexen Kommunikationsnetze realisieren zu können, ohne dass die Teilnehmer detailliertes Wissen über dessen physische Struktur\footnote{Wissen darüber wo sich welcher Repeater befindet und wo sich welche Teilnehemer aufhalten.} benötigen, wurde DMR entwickelt.

\begin{merke}
 DMR hat mehr Ähnlichkeit mit einem Telefonnetz mit zusätzlichen Gruppenruf als mit klassischem FM-Repeaterbetrieb.
\end{merke} 

Das heißt, jeder Teilnehmer und damit dessen Funkgerät besitzt eine eindeutige Nummer\index{DMR-ID}. Diese Nummer liegt im Bereich $1$--$16777215$. Und wie bei einem gewöhnlichen Telefonnetz, kann ein Teilnehmer einen Anderen mit seiner Nummer direkt anrufen. Dies wird \adef{Direktruf} oder auch \altdef{Private Call}{Direktruf} genannt.

Außerdem werden Gruppen definiert, die wieder ihre eigene Nummer erhalten. Die sogenannte \adef{Sprechgruppe} oder auch \altdef{Talk Group}{Sprechgruppe} (\altdef{TG}{Sprechgruppe}). Diese Sprechgruppen dienen dazu, alle Mitarbeiter einer bestimmten Gruppe (z.B., den Wachschutz, die Betriebsfeuerwehr, etc.) gleichzeitig erreichen zu können. Das heißt, das Funkgerät einer Reinigungskraft muss wissen, dass es auf die Gruppenrufe der Sprachgruppe \emph{Reinigung} reagieren muss, aber alle anderen Sprechgruppen ignorieren soll. 

\begin{merke}
 Dieser Punkt ist sehr wichtig: Das DMR Netz selbst weiß nicht, welcher Teilnehmer zu welcher Gruppe gehört. Das Funkgerät des Teilnehmers wird so konfiguriert, dass es nur auf bestimmte Gruppenrufe reagiert.
\end{merke}

\begin{figure}[!ht]
 \centering
 \documentclass{standalone}
\usepackage{tikz}
\usetikzlibrary{shapes.geometric}
\newcommand{\repeater}[3]{%
 \node ({#1}) at ({#2}) {%
  \begin{tikzpicture}%
   \draw [black,thick] (-.25,0) -- (0,0.5) -- (0.25,0) -- (-0.25,0);%
   \draw [black,thick,domain=-45:225] plot ({0.2*cos(\x)}, {0.5+0.2*sin(\x)});%
   \draw [black,thick,domain=-45:225] plot ({0.4*cos(\x)}, {0.5+0.4*sin(\x)});%
   \node (xxx) at (0,-.2) {{#3}};%
  \end{tikzpicture}%
 } %
}

\newcommand{\activerepeater}[3]{%
 \node ({#1}) at ({#2}) {%
  \begin{tikzpicture}%
   \draw [black,thick] (-.25,0) -- (0,0.5) -- (0.25,0) -- (-0.25,0);%
   \draw [red,thick,domain=-45:225] plot ({0.2*cos(\x)}, {0.5+0.2*sin(\x)});%
   \draw [red,thick,domain=-45:225] plot ({0.4*cos(\x)}, {0.5+0.4*sin(\x)});%
   \node (xxx) at (0,-.2) {{#3}};%
  \end{tikzpicture}%
 } %
}


\newcommand{\user}[3]{%
 \node ({#1}) at ({#2}) {%
  \begin{tikzpicture}%
   \draw [black,fill=black] (-.25,0) -- (0,0.5) -- (0.25,0) -- (-0.25,0);%
   \draw [black,fill=black] (0,.5) circle (.2); %
   \node (xxx) [text width=0.6cm, align=center] at (-.35cm,-.4) {{#3}};%
  \end{tikzpicture}%
 } %
}

\newcommand{\activeuser}[3]{%
 \node ({#1}) at ({#2}) {%
  \begin{tikzpicture}%
   \draw [red,fill=red] (-.25,0) -- (0,0.5) -- (0.25,0) -- (-0.25,0);%
   \draw [red,fill=red] (0,.5) circle (.2); %
   \node (xxx) [text width=0.6cm, align=center] at (-.35cm,-.4) {{#3}};%
  \end{tikzpicture}%
 } %
}

\begin{document}
 \begin{tikzpicture}[every node/.style={scale=.8}]
  \user{r1}{ 0,0}{Reinigung 1};
  \user{r2}{ 2,0}{Reinigung 2};	
  \draw[dotted] (3,4) -- (3,-1);
  \user{s1}{ 4,0}{Sicherheit 1};
  \user{z} { 6,0}{Zentrale};
  \draw[dotted] (7,4) -- (7,-1);
  \user{s2}{ 8,0}{Sicherheit 2};
  \user{r3}{10,0}{Reinigung 3};
  \repeater{R1}{1,3}{Terminal 1, TG: R,S};
  \repeater{R2}{5,3}{Terminal 2, TG: R,S};
  \repeater{R3}{9,3}{Vorfeld, TG: S};
 \end{tikzpicture}
\end{document}

 \caption{Ein Beispielnetzwerk für den hypothetischen Flughafen. Es gibt drei Reinigungskräfte, zwei Sicherheitsleute und eine Zentrale. Um das gesammte Gelände abdecken zu können, werden drei Repeater benötigt einer in Terminal 1, einer in Terminal 2 und einer im Vorfeld.} \label{fig:exnet1}
\end{figure}

In Abbildung \ref{fig:exnet1} sei ein Beispielnetzwerk für den Flughafen dargestellt (in Wirklichkeit viel größer und komplexer). Nun stellen wir uns die Situation vor, dass die Reinigungskräfte 1 \& 3 miteinander Sprechen wollen und gleichzeitig die \emph{Zentrale} mit \emph{Sicherheit 1}. In einem einfachen analog Netz, bei dem alle Repeater einfach zusammengeschaltet wären, würde das Gespräch zwischen \emph{Reinigung 1} \& \emph{3} das gesamte Netz blockieren und die Verbindung zwischen \emph{Zentrale} und \emph{Sicherheit 1} wäre nicht möglich. 

\begin{figure}[!ht]
 \centering
 \documentclass{standalone}
\usepackage{tikz}
\usetikzlibrary{shapes.geometric}
\newcommand{\repeater}[3]{%
 \node ({#1}) at ({#2}) {%
  \begin{tikzpicture}%
   \draw [black,thick] (-.25,0) -- (0,0.5) -- (0.25,0) -- (-0.25,0);%
   \draw [black,thick,domain=-45:225] plot ({0.2*cos(\x)}, {0.5+0.2*sin(\x)});%
   \draw [black,thick,domain=-45:225] plot ({0.4*cos(\x)}, {0.5+0.4*sin(\x)});%
   \node (xxx) at (0,-.2) {{#3}};%
  \end{tikzpicture}%
 } %
}

\newcommand{\activerepeater}[3]{%
 \node ({#1}) at ({#2}) {%
  \begin{tikzpicture}%
   \draw [black,thick] (-.25,0) -- (0,0.5) -- (0.25,0) -- (-0.25,0);%
   \draw [red,thick,domain=-45:225] plot ({0.2*cos(\x)}, {0.5+0.2*sin(\x)});%
   \draw [red,thick,domain=-45:225] plot ({0.4*cos(\x)}, {0.5+0.4*sin(\x)});%
   \node (xxx) at (0,-.2) {{#3}};%
  \end{tikzpicture}%
 } %
}


\newcommand{\user}[3]{%
 \node ({#1}) at ({#2}) {%
  \begin{tikzpicture}%
   \draw [black,fill=black] (-.25,0) -- (0,0.5) -- (0.25,0) -- (-0.25,0);%
   \draw [black,fill=black] (0,.5) circle (.2); %
   \node (xxx) [text width=0.6cm, align=center] at (-.35cm,-.4) {{#3}};%
  \end{tikzpicture}%
 } %
}

\newcommand{\activeuser}[3]{%
 \node ({#1}) at ({#2}) {%
  \begin{tikzpicture}%
   \draw [red,fill=red] (-.25,0) -- (0,0.5) -- (0.25,0) -- (-0.25,0);%
   \draw [red,fill=red] (0,.5) circle (.2); %
   \node (xxx) [text width=0.6cm, align=center] at (-.35cm,-.4) {{#3}};%
  \end{tikzpicture}%
 } %
}

\begin{document}
 \begin{tikzpicture}[every node/.style={scale=.8}]
  \activeuser{r1}{ 0,0}{Clean 1};
  \user{r2}{ 2,0}{Clean 2};	
  \draw[dotted] (3,4) -- (3,-1);
  \activeuser{s1}{ 4,0}{Security 1};
  \activeuser{z} { 6,0}{HQ};
  \draw[dotted] (7,4) -- (7,-1);
  \user{s2}{ 8,0}{Security 2};
  \activeuser{r3}{10,0}{Clean 3};
  \activerepeater{R1}{1,3.5}{Terminal 1, TG: C,S};
  \activerepeater{R2}{5,3.5}{Terminal 2, TG: C,S};
  \activerepeater{R3}{9,3.5}{Apron, TG: S};
  \draw[->] (r1) -- node[above,rotate=74] {PC: Clean 3} (R1);
  \path[->,dashed] (R1) edge [bend left] node[above] {via network} (R3);
  \draw[->] (R3) -- node[above,rotate=-74] {PC: Clean 3} (r3);
  \draw[->] (z) -- node[above,rotate=-74] {PC: Security 1} (R2);
  \draw[->] (R2) -- node[above,rotate=74] {PC: Security 1} (s1);
 \end{tikzpicture}
\end{document}

 \caption{Zwei gleichzeitige Direktrufe (Private Calls, PC) in dem Beispielnetzwerk zwischen \emph{Reinigung 1 \& 3} sowie zwischen \emph{Zentrale} und \emph{Sicherheit 1}} \label{fig:exnet2}
\end{figure}

In einem DMR Netz hingegen, werden für einen Direktruf (Privat Call) nur jene Repeater verwendet, die dafür nötig sind. Dies ist in Abbildung \ref{fig:exnet2} zu sehen: \emph{Reinigung 1} startet einen Direktruf (Private Call) über ihren lokalen Repeater in \emph{Terminal 1}. Da das DMR Netzwerk weiß, über welchen Repeater \emph{Reinigung 3} zuletzt aktiv war, wird der Direktruf vom DMR Netz über eben diesen Repeater auf dem Vorfeld etabliert. Der Repeater im Terminal 2 hingegen wird für diesen Direktruf nicht aktiv. Daher steht dieser Repeater weiterhin zur Verfügung. Dies nutzt die Zentrale um \emph{Sicherheit 1} per Direktruf zu erreichen. 

Solange das Gespräch zwischen \emph{Reinigung 1 \& 3} anhält sind aber die Repeater im Terminal 1 und auf dem Vorfeld belegt. Das heißt, die Zentrale kann \emph{Reinigung 2} und \emph{Sicherheit 2} nicht erreichen. Dies klingt schlimmer als es ist. Im Gegensatz zu klassischen Telefonaten gilt im DMR Netz ein Direktruf als unterbrochen sobald ein Teilnehmer die PTT Taste loslässt. Daher kann die Zentrale in den Umschaltpausen des Gespräches \emph{dazwischenrufen} und so zum Beispiel \emph{Sicherheit 2} erreichen. 

Im nächsten Beispiel (Abbildung \ref{fig:exnet3}) will die Zentrale alle Reinigungskräfte erreichen. Dazu macht sie einen Gruppenruf zur Sprechgruppe/Talk Group \emph{Reinigung} (R für Reinigung, S für Sicherheit). Damit erreicht sie die \emph{Reinigung 1 \& 2} problemlos, aber \emph{Reinigung 3} empfängt diesen Gruppenruf nicht. 

Dies liegt daran, dass das DMR Netz nicht weiß, welche Personen zu welcher Gruppe gehören. Da sich Reinigungskräfte üblicherweise nicht auf dem Vorfeld herumtreiben, hat der Repeater auf dem Vorfeld die Sprechgruppe \emph{Reinigung (R)} nicht \emph{abonniert} und leitet daher keine Gruppenrufe für diese Sprechgruppe weiter. 

\begin{figure}[p]
 \begin{subfigure}{\linewidth}
  \centering
  \documentclass{standalone}
\usepackage{tikz}
\usetikzlibrary{shapes.geometric}
\newcommand{\repeater}[3]{%
 \node ({#1}) at ({#2}) {%
  \begin{tikzpicture}%
   \draw [black,thick] (-.25,0) -- (0,0.5) -- (0.25,0) -- (-0.25,0);%
   \draw [black,thick,domain=-45:225] plot ({0.2*cos(\x)}, {0.5+0.2*sin(\x)});%
   \draw [black,thick,domain=-45:225] plot ({0.4*cos(\x)}, {0.5+0.4*sin(\x)});%
   \node (xxx) at (0,-.2) {{#3}};%
  \end{tikzpicture}%
 } %
}

\newcommand{\activerepeater}[3]{%
 \node ({#1}) at ({#2}) {%
  \begin{tikzpicture}%
   \draw [black,thick] (-.25,0) -- (0,0.5) -- (0.25,0) -- (-0.25,0);%
   \draw [red,thick,domain=-45:225] plot ({0.2*cos(\x)}, {0.5+0.2*sin(\x)});%
   \draw [red,thick,domain=-45:225] plot ({0.4*cos(\x)}, {0.5+0.4*sin(\x)});%
   \node (xxx) at (0,-.2) {{#3}};%
  \end{tikzpicture}%
 } %
}


\newcommand{\user}[3]{%
 \node ({#1}) at ({#2}) {%
  \begin{tikzpicture}%
   \draw [black,fill=black] (-.25,0) -- (0,0.5) -- (0.25,0) -- (-0.25,0);%
   \draw [black,fill=black] (0,.5) circle (.2); %
   \node (xxx) [text width=0.6cm, align=center] at (-.35cm,-.4) {{#3}};%
  \end{tikzpicture}%
 } %
}

\newcommand{\activeuser}[3]{%
 \node ({#1}) at ({#2}) {%
  \begin{tikzpicture}%
   \draw [red,fill=red] (-.25,0) -- (0,0.5) -- (0.25,0) -- (-0.25,0);%
   \draw [red,fill=red] (0,.5) circle (.2); %
   \node (xxx) [text width=0.6cm, align=center] at (-.35cm,-.4) {{#3}};%
  \end{tikzpicture}%
 } %
}

\begin{document}
  \begin{tikzpicture}[every node/.style={scale=.8}]
   \activeuser{r1}{ 0,0}{Reinigung 1};
   \activeuser{r2}{ 2,0}{Reinigung 2};	
   \draw[dotted] (3,4) -- (3,-1);
   \user{s1}{ 4,0}{Sicherheit 1};
   \activeuser{z} { 6,0}{Zentrale};
   \draw[dotted] (7,4) -- (7,-1);
   \user{s2}{ 8,0}{Sicherheit 2};
   \user{r3}{10,0}{Reinigung 3};
   \activerepeater{R1}{1,3}{Terminal 1, TG: R,S};
   \activerepeater{R2}{5,3}{Terminal 2, TG: R,S};
   \repeater{R3}{9,3}{Vorfeld, TG: S};
   \draw[->] (z) -- node[above,rotate=-74] {TG: R} (R2);
   \path[->,dashed] (R2) edge [bend right] node[above] {via Netzwerk} (R1);
   \draw[->] (R1) -- node[above,rotate=74] {TG: R} (r1);
   \draw[->] (R1) -- node[above,rotate=-74] {TG: R} (r2);
  \end{tikzpicture}
\end{document}

  \caption{Ein Gruppenruf zur Sprechgruppe \emph{Reinigung} von der Zentrale aus. Der Teilnehmer \emph{Reinigung 3} wird aber nicht erreicht, da der Vorfeldrepeater diese Sprechgruppe nicht abonniert hat.} \label{fig:exnet3}
 \end{subfigure}\vspace{0.5cm}
 \begin{subfigure}{\linewidth}
  \centering
  \documentclass{standalone}
\usepackage{tikz}
\usetikzlibrary{shapes.geometric}
\newcommand{\repeater}[3]{%
 \node ({#1}) at ({#2}) {%
  \begin{tikzpicture}%
   \draw [black,thick] (-.25,0) -- (0,0.5) -- (0.25,0) -- (-0.25,0);%
   \draw [black,thick,domain=-45:225] plot ({0.2*cos(\x)}, {0.5+0.2*sin(\x)});%
   \draw [black,thick,domain=-45:225] plot ({0.4*cos(\x)}, {0.5+0.4*sin(\x)});%
   \node (xxx) at (0,-.2) {{#3}};%
  \end{tikzpicture}%
 } %
}

\newcommand{\activerepeater}[3]{%
 \node ({#1}) at ({#2}) {%
  \begin{tikzpicture}%
   \draw [black,thick] (-.25,0) -- (0,0.5) -- (0.25,0) -- (-0.25,0);%
   \draw [red,thick,domain=-45:225] plot ({0.2*cos(\x)}, {0.5+0.2*sin(\x)});%
   \draw [red,thick,domain=-45:225] plot ({0.4*cos(\x)}, {0.5+0.4*sin(\x)});%
   \node (xxx) at (0,-.2) {{#3}};%
  \end{tikzpicture}%
 } %
}


\newcommand{\user}[3]{%
 \node ({#1}) at ({#2}) {%
  \begin{tikzpicture}%
   \draw [black,fill=black] (-.25,0) -- (0,0.5) -- (0.25,0) -- (-0.25,0);%
   \draw [black,fill=black] (0,.5) circle (.2); %
   \node (xxx) [text width=0.6cm, align=center] at (-.35cm,-.4) {{#3}};%
  \end{tikzpicture}%
 } %
}

\newcommand{\activeuser}[3]{%
 \node ({#1}) at ({#2}) {%
  \begin{tikzpicture}%
   \draw [red,fill=red] (-.25,0) -- (0,0.5) -- (0.25,0) -- (-0.25,0);%
   \draw [red,fill=red] (0,.5) circle (.2); %
   \node (xxx) [text width=0.6cm, align=center] at (-.35cm,-.4) {{#3}};%
  \end{tikzpicture}%
 } %
}

\begin{document}
  \begin{tikzpicture}[every node/.style={scale=.8}]
   \activeuser{r1}{ 0,0}{Reinigung 1};
   \activeuser{r2}{ 2,0}{Reinigung 2};	
   \draw[dotted] (3,4) -- (3,-1);
   \user{s1}{ 4,0}{Sicherheit 1};
   \user{z} { 6,0}{Zentrale};
   \draw[dotted] (7,4) -- (7,-1);
   \user{s2}{ 8,0}{Sicherheit 2};
   \user{r3}{10,0}{Reinigung 3};
   \activerepeater{R1}{1,3}{Terminal 1, TG: R,S};
   \repeater{R2}{5,3}{Terminal 2, TG: R,S};
   \activerepeater{R3}{9,3}{Vorfeld, TG: S,(R)};
   \draw[->] (r3) -- node[above,rotate=-74] {TG: R} (R3);
   \path[->,dashed] (R3) edge [bend right] node[above] {via Netzwerk} (R1);
   \draw[->] (R1) -- node[above,rotate=74] {TG: R} (r1);
   \draw[->] (R1) -- node[above,rotate=-74] {TG: R} (r2);
  \end{tikzpicture}
\end{document}

  \caption{Teilnehmer \emph{Reinigung 3} abonniert die Sprechgruppe \emph{Reinigung} temporär auf dem Vorfeldrepeater, indem er einen Gruppenruf zu dieser Sprechgruppe startet.} \label{fig:exnet4a} 
 \end{subfigure}\vspace{.5cm}
 \begin{subfigure}{\linewidth}
  \centering
  \input{../fig/trunk_net_ex4b}
  \caption{Nach der temporären Abonnierung, ist nun der Teilnehmer \emph{Reinigung 3} auch auf dem Vorfeld erreichbar.} \label{fig:exnet4b}
 \end{subfigure}
 \caption{Temporäres Abonnement einer Sprechgruppe auf einem Repeater.} \label{fig:exnet4}
\end{figure}

Damit die Reinigungskraft 3 jedoch für Gruppenrufe erreichbar bleibt, muss sie die Sprechgruppe \emph{Reinigung} auf dem Vorfeldrepeater temporär abonnieren. Dazu startet sie einen Gruppenruf zur Sprechgruppe \emph{Reinigung} vom Vorfeldrepeater aus (siehe Abb. \ref{fig:exnet4a}). Damit abonniert der Vorfeldrepeater diese Sprechgruppe für eine begrenzte Zeit\footnote{Diese Zeit wird auf jedem einzelnen Repeater konfiguriert. Üblich sind Zeiten zwischen $10$ und $30$ Minuten.} und wird während dieser Zeit Gruppenrufe dieser Sprechgruppe aussenden. 

Dieses temporäre Abonnement wird jedes mal erneuert oder wiederhergestellt, wenn ein Gruppenruf zu dieser Sprechgruppe von diesem Repeater aus initiiert wird. Das heißt, das Abonnement verlängert sich jedes mal, wenn \emph{Reinigung 3} einen Gruppenruf zur Sprechgruppe \emph{Reinigung} startet oder darauf antwortet\footnote{Das Antworten auf einen Gruppenruf ist technisch identisch zum Start eines neuen Gruppenrufs.}.

Mit diesen Beispielen sind die wichtigsten Grundbegriffe von DMR (DMR-ID, Talk Groups, Private sowie Group Call \& Talk Group Abonnement) eingeführt und deren Verwendung in einem Beispiel DMR-Netz erläutert worden. In den nächsten Absätzen wird die Verwendung von DMR im Amateurfunk beschrieben.



\section{DMR Simplex Betrieb} \label{sec:simplex}
Die einfachste Form eines DMR QSOs\footnote{Für alle nicht-Funkamateure: QSO ist eine Abkürzung die eine Verbindung zwischen zwei Amateurfunkstationen beschreibt, gelesen als \emph{Verbindung} oder \emph{Gespräch}.} ist der \aref{Simplexbetrieb}. Dabei wird eine direkte Verbindung zwischen zwei DMR Funkgeräten aufgebaut. Wie beim DMR Repeaterbetrieb, kann so eine Verbindung ein Direktruf, Gruppenruf oder auch ein sogenannter \adef{Rundumruf} (auch \altdef{All Call}{Rundumruf} genannt) sein. 

\begin{figure}[!ht]
 \centering
 \documentclass{standalone}
\usepackage{tikz}
\usetikzlibrary{shapes.geometric}
\newcommand{\repeater}[3]{%
 \node ({#1}) at ({#2}) {%
  \begin{tikzpicture}%
   \draw [black,thick] (-.25,0) -- (0,0.5) -- (0.25,0) -- (-0.25,0);%
   \draw [black,thick,domain=-45:225] plot ({0.2*cos(\x)}, {0.5+0.2*sin(\x)});%
   \draw [black,thick,domain=-45:225] plot ({0.4*cos(\x)}, {0.5+0.4*sin(\x)});%
   \node (xxx) at (0,-.2) {{#3}};%
  \end{tikzpicture}%
 } %
}

\newcommand{\activerepeater}[3]{%
 \node ({#1}) at ({#2}) {%
  \begin{tikzpicture}%
   \draw [black,thick] (-.25,0) -- (0,0.5) -- (0.25,0) -- (-0.25,0);%
   \draw [red,thick,domain=-45:225] plot ({0.2*cos(\x)}, {0.5+0.2*sin(\x)});%
   \draw [red,thick,domain=-45:225] plot ({0.4*cos(\x)}, {0.5+0.4*sin(\x)});%
   \node (xxx) at (0,-.2) {{#3}};%
  \end{tikzpicture}%
 } %
}


\newcommand{\user}[3]{%
 \node ({#1}) at ({#2}) {%
  \begin{tikzpicture}%
   \draw [black,fill=black] (-.25,0) -- (0,0.5) -- (0.25,0) -- (-0.25,0);%
   \draw [black,fill=black] (0,.5) circle (.2); %
   \node (xxx) [text width=0.6cm, align=center] at (-.35cm,-.4) {{#3}};%
  \end{tikzpicture}%
 } %
}

\newcommand{\activeuser}[3]{%
 \node ({#1}) at ({#2}) {%
  \begin{tikzpicture}%
   \draw [red,fill=red] (-.25,0) -- (0,0.5) -- (0.25,0) -- (-0.25,0);%
   \draw [red,fill=red] (0,.5) circle (.2); %
   \node (xxx) [text width=0.6cm, align=center] at (-.35cm,-.4) {{#3}};%
  \end{tikzpicture}%
 } %
}

\begin{document}
 \begin{tikzpicture}[every node/.style={scale=.8}]
  \activeuser{u1}{ 0,0}{DM3MAT};
  \activeuser{u2}{ 6,1}{DL1XYZ, TG99};
  \user{u3}{ 6,0}{DL2XYZ, TG99};
  \user{u4}{ 6,-1}{DL3XYZ};
  \path[->] (u1) edge[bend left] node[above, rotate=10]{$433.450 MHz$} node[below, rotate=10]{PC: DL1XYZ} (u2);
  \path[->] (u2) edge[bend left] node[above, rotate=10]{$433.450 MHz$} node[below, rotate=10]{PC: DM3MAT} (u1);
 \end{tikzpicture}
\end{document}

 \caption{Beispiel eines DMR Simplex Direktrufs von DM3MAT an DL1XYZ.} \label{fig:splxpc}
\end{figure}

In Abbildung \ref{fig:splxpc} ist ein einfacher Simplex Direktruf von DM3MAT an DL1XYZ dargestellt sowie dessen Antwort. Beide senden und empfangen auf der selben Frequenz (hier der DMR Anruffrequenz von $433.450 MHz$). Auch wenn die beiden anderen Teilnehmer in der Nähe (DL2XYZ \& DL3XYZ) diesen Ruf physikalisch empfangen, bleiben deren Funkgeräte stumm. Wie dem auch sei, der Kanal ist jedoch während dieses Direktrufes belegt. 

An dieser Stelle ist es sinnvoll zu erwähnen, dass wenn DL1XYZ direkt auf den Direktruf von DM3MAT antwortet, indem er die PTT Taste drückt, er mit einem Direktruf an DM3MAT antwortet, ohne dafür die Nummer von DM3MAT aus seinen Kontakten heraussuchen zu müssen. Diese Eigenschaft heißt \adef{Talkaround} und funktioniert wenige Sekunden nach dem Ende des initialen Direktrufs durch DM3MAT. Nach dieser Zeitspanne wird beim drücken auf die PTT der \aref{Standardkontakt} für diesen Kanal angerufen, der für jeden Kanal im Funkgerät festgelegt werden kann (siehe Abs. \ref{sec:cp:channel}). Diese Zeitspanne (genannt \aref{Hangtime}) lässt sich im Funkgerät einstellen.

\begin{figure}[!ht]
  \centering
  \documentclass{standalone}
\usepackage{tikz}
\usetikzlibrary{shapes.geometric}
\newcommand{\repeater}[3]{%
 \node ({#1}) at ({#2}) {%
  \begin{tikzpicture}%
   \draw [black,thick] (-.25,0) -- (0,0.5) -- (0.25,0) -- (-0.25,0);%
   \draw [black,thick,domain=-45:225] plot ({0.2*cos(\x)}, {0.5+0.2*sin(\x)});%
   \draw [black,thick,domain=-45:225] plot ({0.4*cos(\x)}, {0.5+0.4*sin(\x)});%
   \node (xxx) at (0,-.2) {{#3}};%
  \end{tikzpicture}%
 } %
}

\newcommand{\activerepeater}[3]{%
 \node ({#1}) at ({#2}) {%
  \begin{tikzpicture}%
   \draw [black,thick] (-.25,0) -- (0,0.5) -- (0.25,0) -- (-0.25,0);%
   \draw [red,thick,domain=-45:225] plot ({0.2*cos(\x)}, {0.5+0.2*sin(\x)});%
   \draw [red,thick,domain=-45:225] plot ({0.4*cos(\x)}, {0.5+0.4*sin(\x)});%
   \node (xxx) at (0,-.2) {{#3}};%
  \end{tikzpicture}%
 } %
}


\newcommand{\user}[3]{%
 \node ({#1}) at ({#2}) {%
  \begin{tikzpicture}%
   \draw [black,fill=black] (-.25,0) -- (0,0.5) -- (0.25,0) -- (-0.25,0);%
   \draw [black,fill=black] (0,.5) circle (.2); %
   \node (xxx) [text width=0.6cm, align=center] at (-.35cm,-.4) {{#3}};%
  \end{tikzpicture}%
 } %
}

\newcommand{\activeuser}[3]{%
 \node ({#1}) at ({#2}) {%
  \begin{tikzpicture}%
   \draw [red,fill=red] (-.25,0) -- (0,0.5) -- (0.25,0) -- (-0.25,0);%
   \draw [red,fill=red] (0,.5) circle (.2); %
   \node (xxx) [text width=0.6cm, align=center] at (-.35cm,-.4) {{#3}};%
  \end{tikzpicture}%
 } %
}

\begin{document}
  \begin{tikzpicture}[every node/.style={scale=.8}]
   \activeuser{u1}{ 0,0}{DM3MAT};
   \user{u2}{ 6,2}{DL1XYZ, TG99};
   \user{u3}{ 6,0}{DL2XYZ, TG99};
   \user{u4}{ 6,-2}{DL3XYZ};
   \path[->] (u1) edge[bend left] node[above, rotate=10]{$433.450 MHz$} node[below, rotate=10]{GC: TG99} (u2);
   \path[->] (u1) edge node[above]{$433.450 MHz$} node[below]{GC: TG99} (u3);
  \end{tikzpicture}
\end{document}

  \caption{Beispiel eines DMR Simplex Gruppenrufs von DM3MAT an die Sprechgruppe TG99.} \label{fig:splxgc}
\end{figure}

Um im Simplexbetrieb nicht nur einzelne Teilnehmer anrufen zu können, sind auch Gruppenrufe im Simplexbetrieb möglich. Eine beliebte Sprechgruppe (Talk Group) für den Simplexbetrieb ist die Gruppe mit der Nummer 99  (TG99 abgekürzt, für \emph{Talk Group 99}). Solche Gruppenrufe werden dann von allen Funkgeräten empfangen, die entsprechend konfiguriert wurden. Wie beim Repeaterbetrieb muss auch beim Simplexbetrieb dem Funkgerät mitgeteilt werden, welche Sprechgruppen es auf welchen Kanälen empfangen soll (siehe Abs. \ref{sec:cp:grouplist}). 

In Abbildung \ref{fig:splxgc} ist solch ein Simplex Gruppenruf von DM3MAT an die Sprechgruppe TG99 dargestellt. Da DL1XYZ und DL2XYZ ihre Funkgeräte so konfiguriert haben, dass sie die TG99 empfangen, hören sie den Ruf von DM3MAT. Da DL3XYZ dies nicht gemacht hat, empfängt er diesen Ruf nicht. DL1XYZ und DL2XYZ können nun auf diesen Gruppenruf antworten, wenn sie innerhalb der sogenannten \adef{Haltezeit} (\altdef{Hangtime}{Haltezeit}) auf ihre PTT Taste drücken. Sie würden dann ebenfalls mit einem Gruppenruf zur TG99 antworten (Talkaround gilt auch für Gruppenrufe), auch wenn sie einen anderen Standardkontakt für diesen Simplexkanal eingestellt haben.

\begin{figure}[!ht]
  \centering
  \input{../fig/simplex_allcall}
  \caption{Beispiel eines DMR All Calls von DM3MAT alle die ihn hören können.} \label{fig:splxac}
\end{figure}

Um wirklich sicher zu gehen, dass ein Ruf auf einem Simplexkanal von allen empfangen werden kann, sollte ein sogenannter \adef{All Call} verwendet werden. Dieser Ruf ist ein spezieller Ruf an eine ganz bestimmte Nummer ($16777215$), die von allen Geräten empfangen werden unabhängig von der Konfiguration dieser Geräte. In diesem Beispiel wird somit der Ruf von DM3MAT auch von DL3XYZ empfangen. Durch \adef{Talkaround} ist es allen Teilnehmern wieder möglich auf den All Call von DM3MAT zu antworten, auch wenn diese Teilnehmer den All Call nicht als den Standardkontakt für diesen Kanal konfiguriert haben. 

\begin{merke}
 Zusammengefasst: Ein DMR-Kanal besitzt eine Sende- \& Empfangsfrequenz (bei Simplex identisch), einen Standardkontakt der angerufen wird, wenn die PTT Taste auf diesem Kanal gedrückt wird und eine Liste von Gruppenrufen, die auf diesem Kanal empfangen werden sollen. 
\end{merke}

\subsection{DMR Simplex Frequenzen}
\begin{table}[!ht]
 \centering
 \begin{tabular}{|l|c||l|c|} \hline
  Name & Frequenz & Name & Frequenz \\ \hline \hline
  S0 (Anruf) & $433.4500 MHz$ & S4 & $433.6500 MHz$ \\
  S1         & $433.6125 MHz$ & S5 & $433.6625 MHz$ \\
  S2         & $433.6250 MHz$ & S6 & $433.6750 MHz$ \\
  S3         & $433.6375 MHz$ & S7 & $433.6875 MHz$ \\ \hline
 \end{tabular}
 \caption{Liste der acht üblichen DMR Simplexkanäle. Der Kanal \emph{S0} ist der Anrufkanal.} \label{tab:simplex}
\end{table}

Im Tabelle \ref{tab:simplex} sind die acht üblichen Simplexkanäle aufgelistet. Der Simplexkanal \emph{S0} ist dabei der Anrufkanal. Gerade in Ballungsgebieten sollte für das eigentliche QSO der Kanal vom Anrufkanal auf einen der sieben weiteren Simplexkanäle \emph{S1-7} gewechselt werden, um den Anrufkanal nicht zu blockieren. 

\section{Lokaler Repeater Betrieb} \label{sec:lokal}	

\section{Direkte Anrufe} \label{sec:privatecall}

\section{Datendienste} \label{sec:data}
Da DMR von sich aus schon eine digitale Betriebsart ist, bei der meist Sprache in digitalisierter 
Form übertragen wird, ist es natürlich auch möglich reine Datendienste über DMR anzubieten. Zum 
einen gibt es einen Textnachrichtendienst, der dem SMS-Dienst der Mobiltelefone nachempfunden ist. 
Zum anderen gibt es auch die Möglichkeit, die eigene Position per DMR an das APRS\footnote{APRS 
steht für \emph{Automatic Packet Reporting System} und ermöglicht das Übertragen von kleinen 
Datensätzen über Packet-Radio wie zum Beispiel die Position, Wetter oder Textnachrichten. Mehr 
dazu erfahren sie in der \href{https://de.wikipedia.org/wiki/Automatic_Packet_Reporting_System}{Wikipedia}.} 
Netz zu übertragen.
 
\subsection{Textnachrichten (SMS)} \label{sec:textmsg}
Mit diesem Dienst können sie kurze Textnachrichten\footnote{Bis zu 144 Zeichen.} direkt an andere Teilnehmer verschicken\footnote{Sie können auch Textnachrichten an ganze Sprechgruppen versenden. Dies ist aber eher unüblich und nicht wünschenswert.}. Im Prinzip funktioniert eine Textnachricht wie ein Direktruf. Ist der andere Teilnehmer erreichbar, wird die Textnachricht übermittelt. 

Es gibt aber auch \emph{Servicenummern} (gebührenfrei). Wenn sie nun eine Nachricht an eine solche Nummer senden, können Sie bestimmte Informationen abrufen oder versenden. In Deutschland wären das:
\begin{enumerate}
 \item 262993 -- GPS und Wetter
 \begin{itemize}
  \item Wenn Sie \texttt{help} senden, erhalten daraufhin eine Auflistung aller Kommandos.
  \item Wenn Sie \texttt{wx} senden, erhalten Sie das aktuelle Wetter am Standort des Repeaters, den Sie verwenden.
  \item Wenn Sie \texttt{wx STADTNAME} senden, erhalten Sie das aktuelle Wetter für die angegebene Stadt.
  \item Wenn Sie \texttt{gps} senden, erhalten Sie die letzte Positionsinformation, die Sie zuletzt an das DMR Netz gesendet hatten.
  \item Mit \texttt{gps CALL} können Sie auch die letzte Position des angegebenen Teilnehmers abfragen.
  \item Mit \texttt{rssi} erhalten Sie vom Repeater einen Signalrapport.
 \end{itemize} 
 \item 262994 -- Repeater Informationen \& Pagernachrichten
  \begin{itemize}
   \item Wenn Sie \texttt{rpt} senden, erhalten Sie eine Liste der statisch und dynamisch abonnierten Sprechgruppen des Repeaters.  
   \item Wenn Sie \texttt{CALL NACHRICHT}, wird die angegebene Nachricht an das angegebene Call per Pager (DAPNET) geschickt.
  \end{itemize}
\end{enumerate}
 
\subsection{Positionsübermittlung (APRS via DMR)} \label{sec:aprs}
Wie im vorherigen Abschnitt schon erwähnt, ist es möglich seine Position ins DMR Netz zu senden. Diese wird dann üblicherweise direkt an das APRS-Netz weitergereicht und Ihre Position kann dann unter anderem bei \url{https://aprs.fi} abgefragt werden. Dazu ist jedoch ein DMR Funkgerät mit GPS Empfänger nötig. Aber auch diese Geräte sind in der Zwischenzeit nicht mehr teuer. Einfache DMR Handfunkgeräte mit GPS sind ab circa \EUR{120} zu haben. 

Neben dem SMS Service ist auch die Positionsübermitellung per DMR möglich. Dazu muss das GPS fähige Funkgerät so konfiguriert werden, dass die Positionsdaten auf den geeigneten Kanälen an die Nummer 262999 gesendet werden. Wie dies einzustellen ist, hängt sehr vom Hersteller des Funkgerätes ab. 

\section{Talkgroup Betrieb} \label{sec:talkgroup}
Ein klassisches und auch schönes Beispiel für den Repeater-transparenten Sprechgruppenbetrieb (Talkgroup) ist das Szenario einer Nachmittagsrunde in einer Sprechgruppe. Zum Beispiel, die Sprechgruppe 2621 \emph{Berlin/Brandenburg} kurz BB. Diese Sprechgruppe ist bei fast allen Repeatern in Berlin und Brandenburg fest Abonniert. Das heißt, diese Runde kann ohne weiteres Zutun in ganz Berlin und Brandenburg empfangen werden (siehe Abb. \ref{fig:tgex1}). 

\begin{figure}[p]
 \centering
 \begin{subfigure}{\linewidth}
  \centering
  \input{../fig/talkgroup_ex1a}
  \caption{Beispiel für eine typische Nachmittagsrunde auf einer Sprechgruppe.} \label{fig:tgex1}
 \end{subfigure}
 \begin{subfigure}{\linewidth}
  \centering
  \documentclass{standalone}
\usepackage{tikz}
\usetikzlibrary{shapes.geometric}
\newcommand{\repeater}[3]{%
 \node ({#1}) at ({#2}) {%
  \begin{tikzpicture}%
   \draw [black,thick] (-.25,0) -- (0,0.5) -- (0.25,0) -- (-0.25,0);%
   \draw [black,thick,domain=-45:225] plot ({0.2*cos(\x)}, {0.5+0.2*sin(\x)});%
   \draw [black,thick,domain=-45:225] plot ({0.4*cos(\x)}, {0.5+0.4*sin(\x)});%
   \node (xxx) at (0,-.2) {{#3}};%
  \end{tikzpicture}%
 } %
}

\newcommand{\activerepeater}[3]{%
 \node ({#1}) at ({#2}) {%
  \begin{tikzpicture}%
   \draw [black,thick] (-.25,0) -- (0,0.5) -- (0.25,0) -- (-0.25,0);%
   \draw [red,thick,domain=-45:225] plot ({0.2*cos(\x)}, {0.5+0.2*sin(\x)});%
   \draw [red,thick,domain=-45:225] plot ({0.4*cos(\x)}, {0.5+0.4*sin(\x)});%
   \node (xxx) at (0,-.2) {{#3}};%
  \end{tikzpicture}%
 } %
}


\newcommand{\user}[3]{%
 \node ({#1}) at ({#2}) {%
  \begin{tikzpicture}%
   \draw [black,fill=black] (-.25,0) -- (0,0.5) -- (0.25,0) -- (-0.25,0);%
   \draw [black,fill=black] (0,.5) circle (.2); %
   \node (xxx) [text width=0.6cm, align=center] at (-.35cm,-.4) {{#3}};%
  \end{tikzpicture}%
 } %
}

\newcommand{\activeuser}[3]{%
 \node ({#1}) at ({#2}) {%
  \begin{tikzpicture}%
   \draw [red,fill=red] (-.25,0) -- (0,0.5) -- (0.25,0) -- (-0.25,0);%
   \draw [red,fill=red] (0,.5) circle (.2); %
   \node (xxx) [text width=0.6cm, align=center] at (-.35cm,-.4) {{#3}};%
  \end{tikzpicture}%
 } %
}

\begin{document}
  \begin{tikzpicture}[every node/.style={scale=.8}]
   \user{u1}{ 0,0}{DM3MAT};
   \user{u2}{ 2,0}{DL1XYZ 2};	
   \user{u3}{ 6,0}{DL2XYZ 2};	
   \draw[dotted] (7,4) -- (7,-1);
   \activeuser{u4}{10,0}{I/DL3XYZ};
   \activerepeater{R1}{1,3}{DB0ABC, TG2621};
   \activerepeater{R2}{5,3}{DB0DEF, TG2621};
   \activerepeater{R3}{9,3}{I0ABC, (TG2621)};
   \path[->] (u4) edge node[above,rotate=-70]{GC: TG2621} (R3);
   \path[->] (R1) edge node[above,rotate=70]{GC: TG2621} (u1);
   \path[->] (R1) edge node[above,rotate=-70]{GC: TG2621} (u2);
   \path[->] (R2) edge node[above,rotate=-70]{GC: TG2621} (u3);
   \path[->] (R3) edge[bend right] node[below]{GC: TG2621} (R2);
   \path[->] (R3) edge[bend right] node[above]{GC: TG2621} (R1);
  \end{tikzpicture}
\end{document}

  \caption{Der OM im Ausland abonniert diese Sprechgruppe an dem lokalen Repeater temporär durch einen Gruppenruf zu dieser Sprechgruppe.} \label{fig:tgex2}
 \end{subfigure}
 \begin{subfigure}{\linewidth}
  \centering
  \input{../fig/talkgroup_ex1c}
  \caption{Danach kann auch der OM im Urlaub wie gewohnt an dieser Nachmittagsrunde teilnehmen.} \label{fig:tgex3}
 \end{subfigure}
\end{figure}

Für einen OM im Urlaub, gilt das natürlich nicht. Ein italienischer Repeater wird sicher nicht standardmäßig die Sprechgruppe \emph{Berlin/Brandenburg} abonniert haben. Daher wird dieser OM die Sprechgruppe im Ausland auch nicht hören. Da er aber weiß, wann diese Runde beginnt, kann er vorher per Gruppenruf zu dieser Sprechgruppe von seinem Urlaubsrepeater (I0ABC) aus, diese Sprechgruppe temporär abonnieren (Abb. \ref{fig:tgex2}). 

Nachdem er diese Sprechgruppe beim Urlaubsrepeater abonniert hat, kann er wie gewohnt an der Nachmittagsrunde teilnehmen (Abb. \ref{fig:tgex3}). Für die anderen Teilnehmer dieser Runde ist dann nicht einmal ersichtlich, dass der Urlauber nicht über ein Relais in Berlin oder Brandenburg sondern aus dem Ausland an der Runde teilnimmt.

\begin{table}[!ht]
 \centering
 \begin{tabular}{|l|c|} \hline
  Name & Sprechgruppe \\ \hline
  Global & 91 \\
  Europa & 92 \\
  Deutschland & 262 \\
  Mecklenburg-Vorpommern \& Sachsen-Anhalt & 2620 \\
  Berlin \& Brandenburg & 2621 \\
  Hamburg \& Schleswig-Holstein & 2622 \\
  Niedersachsen \& Bremen & 2623 \\
  Nordrhein-Westfalen & 2624 \\
  Rheinland-Pfalz \& Saarland & 2625 \\
  Hessen & 2626 \\
  Baden-Württemberg & 2627 \\
  Bayern & 2628 \\
  Sachsen \& Thüringen & 2629 \\ \hline
 \end{tabular}
 \caption{} \label{tab:talkgroups}
\end{table}

\subsection{Cluster}
Im Gegensatz zur dediziert Regionalen Gruppe TG8, sind gewöhnliche Sprechgruppen von überall aus dem DMR Netz erreichbar. Das heißt, ein OM der sich gerade im Urlaub befindet und an einer Runde in dieser Sprechgruppe teilnehmen möchte, kann dies wie oben beschrieben tun. 

Würde diese Nachmittagsrunde aber auf der regionalen Sprechgruppe TG8 stattfinden, könnte der OM im Urlaub nicht daran teilnehmen. Er würde an seinem Urlaubsrepeater mit einem Gruppenruf zur Sprechgruppe TG8 nur Funkamateure in seiner Urlaubsregion erreichen aber nicht den regionalen Verbund von Repeatern zu Hause.

Aus diesem Grund werden häufig regionale Verbünde von Repeatern mit sogenannten \emph{Clustern} verbunden. Diese \adef{Cluster} stellen dann eine weitere Sprechgruppennummer für den regionalen Verbund zur Verfügung, sodass die Sprechgruppe TG8 einer bestimmten Region auch von außen erreichbar ist. Eine Liste der Regionalcluster und der dazugehörigen Sprechgruppennummer kann unter \url{http://bm262.de/cluster/} abgerufen werden.


\section{Positionsübermittlung} \label{sec:aprs}

\section{Roaming} \label{sec:roaming}
TODO: Das kann mir ja jemand beim AJW Seminar erklären.

\section{DMR-Netze} \label{sec:netze}
TODO

\subsection{Brandmeister Netz}
TODO

\subsection{DMR+ Netz}
TODO

\subsection{DMR-MARC Netz}
TODO

\section{Technischer Hintergrund} \label{sec:technik}
Nachdem ich in den vorherigen Abschnitten versucht habe Ihnen die Konzepte von DMR (Repeater-unabhängige Direkt und Gruppenrufe) näherzubringen, geht es in diesem Abschnitt an das Eingemachte. Da heißt, die technischen Details und Besonderheiten von DMR. Im speziellen um die Begriffe \emph{Time Slot} und \emph{Color Code}.

\subsection{Zeitschlitze (Time Slots)} \section{sec:timeslot}
Wie zu Beginn erwähnt, ist DMR eine digitale Übertragungstechnik, bei der Sprache zunächst digitalisiert wird, mit einem sogenannten Codec komprimiert wird und als Datenpakete übertragen werden. Modere Sprachcodecs sind in der Zwischenzeit so effizient geworden, dass es möglich ist auf einem $12.5 kHz$ breiten Kanal zwei Sprachsignale in guter Qualität gleichzeitig übertragen zu können. Dies wird auch bei DMR ausgenutzt. DMR verwendet dazu ein Verfahren das sich \adef{TDMA} nennt. 

\begin{figure}[!ht]
 \centering
 \documentclass{standalone}
\usepackage{tikz}
\usetikzlibrary{shapes.geometric}
\usetikzlibrary{patterns,snakes}
\newcommand{\repeater}[3]{%
 \node ({#1}) at ({#2}) {%
  \begin{tikzpicture}%
   \draw [black,thick] (-.25,0) -- (0,0.5) -- (0.25,0) -- (-0.25,0);%
   \draw [black,thick,domain=-45:225] plot ({0.2*cos(\x)}, {0.5+0.2*sin(\x)});%
   \draw [black,thick,domain=-45:225] plot ({0.4*cos(\x)}, {0.5+0.4*sin(\x)});%
   \node (xxx) at (0,-.2) {{#3}};%
  \end{tikzpicture}%
 } %
}

\newcommand{\activerepeater}[3]{%
 \node ({#1}) at ({#2}) {%
  \begin{tikzpicture}%
   \draw [black,thick] (-.25,0) -- (0,0.5) -- (0.25,0) -- (-0.25,0);%
   \draw [red,thick,domain=-45:225] plot ({0.2*cos(\x)}, {0.5+0.2*sin(\x)});%
   \draw [red,thick,domain=-45:225] plot ({0.4*cos(\x)}, {0.5+0.4*sin(\x)});%
   \node (xxx) at (0,-.2) {{#3}};%
  \end{tikzpicture}%
 } %
}


\newcommand{\user}[3]{%
 \node ({#1}) at ({#2}) {%
  \begin{tikzpicture}%
   \draw [black,fill=black] (-.25,0) -- (0,0.5) -- (0.25,0) -- (-0.25,0);%
   \draw [black,fill=black] (0,.5) circle (.2); %
   \node (xxx) [text width=0.6cm, align=center] at (-.35cm,-.4) {{#3}};%
  \end{tikzpicture}%
 } %
}

\newcommand{\activeuser}[3]{%
 \node ({#1}) at ({#2}) {%
  \begin{tikzpicture}%
   \draw [red,fill=red] (-.25,0) -- (0,0.5) -- (0.25,0) -- (-0.25,0);%
   \draw [red,fill=red] (0,.5) circle (.2); %
   \node (xxx) [text width=0.6cm, align=center] at (-.35cm,-.4) {{#3}};%
  \end{tikzpicture}%
 } %
}

\begin{document}
 \begin{tikzpicture}
  \draw[|-,dotted, semithick] (-1,-0.2) -- (0,-0.2);
  \draw[|-,semithick] (0,-0.2) -- (1,-0.2);
  \draw[|-,semithick] (1,-0.2) -- (2,-0.2);
  \draw[|-,semithick] (2,-0.2) -- (3,-0.2);
  \draw[|-,semithick] (3,-0.2) -- (4,-0.2);
  \draw[|-,semithick] (4,-0.2) -- (5,-0.2);
  \draw[|-,semithick] (5,-0.2) -- (6,-0.2);
  \draw[|->,dotted,semithick] (6,-0.2) -- (7,-0.2);
  \node at (7, -.5) {$t$};
  \draw [thick,decoration={brace,mirror},decorate] (0,-0.4) -- (1,-0.4) node [pos=0.5, anchor=north,yshift=-0.55] {$30\ ms$}; 
  \fill[red!30] (0.1,0) -- (0.1,1) -- (0.9,1) -- (0.9,0) -- cycle;
  \node at (0.5,0.5) {TS 1};
  \fill[blue!30] (1.1,0) -- (1.1,1) -- (1.9,1) -- (1.9,0) -- cycle;
  \node at (1.5,0.5) {TS 2};
  \fill[red!30] (2.1,0) -- (2.1,1) -- (2.9,1) -- (2.9,0) -- cycle;
  \node at (2.5,0.5) {TS 1};
  \fill[blue!30] (3.1,0) -- (3.1,1) -- (3.9,1) -- (3.9,0) -- cycle;
  \node at (3.5,0.5) {TS 2};
  \fill[red!30] (4.1,0) -- (4.1,1) -- (4.9,1) -- (4.9,0) -- cycle;
  \node at (4.5,0.5) {TS 1};
  \fill[blue!30] (5.1,0) -- (5.1,1) -- (5.9,1) -- (5.9,0) -- cycle;
  \node at (5.5,0.5) {TS 1};  
 \end{tikzpicture}
\end{document}

 \caption{Graphische Darstellung der \emph{time-division multiple access} (TDMA) Technik.}
\end{figure}

Das steht für \emph{time-division multiple access} beschreibt, wie zwei Teilnehmer (quasi) gleichzeitig einen physischen Kanal (also eine Frequenz) benutzen können. Dazu wird jedem der beiden ein Zeitschlitz zu geordnet (Zeitschlitz 1 und 2) und beide senden oder empfangen nur in ihrem eigenen Zeitschlitz. Diese Zeitschlitze sind sehr kurz, bei DMR nur $30ms$ lang. Diese kurze Zeit reicht jedoch aus um $60ms$ lange Sprachfetzen komprimiert zu übertragen. DMR erhält dadurch zwei völlig unabhängige Kanäle pro Frequenz. Das Bedeutet auch, dass zwei völlig unabhängige Gespräche über einen Repeater gleichzeitig laufen können.

Was oder besser wann nun Zeitschlitz 1 oder 2 dran sind, legt der Repeater fest. Er gibt den Takt vor. Das bedeutet auch, dass Zeitschlitze für den Simplexbetrieb völlig unbedeutend sind. Wenn Sie später einen Simplexkanal für Ihr Funkgerät konfigurieren, ist die Zeitschlitzeinstellung egal.

Was auf welchem Zeitschlitz passieren soll, hängt stark von der Konfiguration des einzelnen Repeaters ab. Grundsätzlich gilt aber:
\begin{merke}
 Überregionale Kommunikation sollte auf Zeitschlitz 1 und lokale sowie regionale Kommunikation auf dem Zeitschlitz 2 stattfinden.
\end{merke} 

\subsection{Farbcodes (Color Codes)} \label{sec:colorcode} \index{Color Code}
Farbcodes sind ein technisches Hilfsmittel, das Störungen zwischen Repeatern vermeiden soll, die auf der selben Frequenz arbeiten. Dieses Problem tritt vor allem im professionellen Einsatz von DMR auf. Einem Unternehmen werden üblicherweise nur wenige Frequenzen zugewiesen, es werden mit unter aber viele Repeater benötigt um ein großes Firmengelände vollständig abdecken zu können (denken Sie an das Flughafenbeispiel). Da bleibt es nicht aus, dass verschiedenen Repeatern die selbe Frequenz zugewiesen werden muss. Wenn sich dann die Reichweiten dieser Repeater überlappen, kann es sein, dass die Aussendungen eines Teilnehmers von zwei Repeatern gleichzeitig aufgenommen werden. Um dies zu verhindern, werden den Repeatern verschiedene sogenannte Farbcodes zugewiesen. Diese kleine zusätzliche Information einer Aussendung erlaubt es einem Repeater oder jedem anderen Teilnehmer zu erkennen, ob eine Aussendung für sie bestimmt ist oder nicht. Nur wenn der Farbcode übereinstimmt, reagiert der Repeater oder das Funkgerät auf diese Aussendung. 

\begin{merke}
 Um einen Repeater nutzen zu können muss nicht nur dessen Ein- und Ausgabefrequenz sondern auch dessen Farbcode bekannt sein!
\end{merke}

\section{Codeplug Programmierung} \label{sec:codeplug}
Nachdem Sie sich mit den Konzepten und dem technischen Hintergrund von DMR auseinandergesetzt haben, geht es nun an die Konfiguration Ihres Funkgerätes. Dies geschieht üblicherweise nicht über das Bedienfeld des Funkgerätes, sonder mit Hilfe einer separaten Software, der sogenannten \adef{CPS} oder \emph{codeplug programming software}. 

Doch bevor Sie loslegen können benötigen Sie wie alle DMR Teilnehmer eine eindeutige Nummer, die DMR ID.
\begin{hinweis}
 Ihre persönliche und eindeutige DMR ID erhalten Sie unter \url{https://register.ham-digital.org/}. Da Sie nachweisen müssen, dass Sie lizenzierter Funkamateur sind, müssen Sie bei der Anmeldung ihre eingescannte \emph{Zulassung zum Amateurfunkdienst} hochladen.
\end{hinweis}
Diese erhalten Sie in der Regel innerhalb von 24 Stunden per Mail. Sobald Sie eine DMR ID erhalten kann es los gehen.

Da dieses Script für Einsteiger gedacht ist, ist es wahrscheinlich das Sie kein top-shelf Motorola Gerät, sonder eher ein günstiges Gerät der einschlägig bekannten chinesischen Hersteller besitzen. 

\begin{achtung}
 Falls Sie noch kein DMR fähiges Funkgerät besitzen und mit dem Gedanken spielen eines zu kaufen, achten Sie unbedingt darauf, dass es DMR \textbf{Tier I \& II}\footnote{Wie so häufig ist DMR nicht ein Standard sondern eine ganze Familie von aufeinander aufbauenden Standards. DMR Tier I beschreibt im wesentlichen den DMR Simplexbetrieb und Tier II dann den Repeaterbetrieb mit zwei Zeitschlitzen. Sie benötigen also unbedingt Tier II für den Repeaterbetrieb.} unterstützt. Ignorieren Sie etwaiges Marketing-Bla-Bla der Hersteller und schauen Sie in den technischen Details nach ob dort DMR \textbf{Tier I \& II} erwähnt wird. Falls nicht oder nicht eindeutig, lassen Sie die Finger von diesem Gerät! Dies gilt vor allem für das Baofeng BR-5R aber nicht für das Baofeng/Radioddity RD5-R\footnote{Manchmal sind es die kleinen Unterschiede die entscheidend sind.}. 
\end{achtung}

Der Hersteller Ihres Gerätes wird auf seiner Webseite die Software, die Sie zur Konfiguration benötigen, zum Download bereitstellen. Diese Software wird \emph{CPS} oder \emph{codeplug programming software} genannt. Gegebenenfalls finden Sie dort auch Firmwareupdates für Ihr Gerät. Viele Hersteller bieten für jedes einzelne Modell eine separate CPS an oder gar für jede Variation eines Modells. Achten Sie also genau darauf welche CPS Sie herunterladen. Die Konfiguration dieser Geräte unterscheidet sich von Gerät zu Gerät und mehr noch von Hersteller zu Hersteller. Jedoch sind die wesentlichen Einstellungen für Geräte dieser Klasse sehr ähnlich.

Wenn Sie die CPS zum ersten mal starten, werden Sie wahrscheinlich zwei Dinge feststellen. Erstens, das Bedienkonzept dieser Software ist aus dem letzten Jahrtausend (Windows 3.11) und Zweitens, es gibt eine Unmenge an obskuren Optionen deren Funktion nicht ersichtlich ist und die größtenteils nicht Dokumentiert sind. Wenn Sie des Englischen nicht mächtig sind, werden Sie auch eine deutsche Übersetzung des Programms vermissen. Aber keine sorge, die englische Übersetzung ist meist auch so schlecht, dass es keinen Unterschied macht ob sie Englisch lesen können oder nicht.

Die Konfiguration Ihres Funkgerätes erfolgt in 5-6 Schritten:
\begin{enumerate}
 \item Allgemeine Einstellungen,
 \item Kontakte anlegen,
 \item Empfangsgruppen festlegen,
 \item alle Kanäle anlegen,
 \item Kanäle in Zonen einteilen und
 \item optional Scanlisten anlegen.
\end{enumerate}

In den folgenden Abschnitten möchte ich die einzelnen Konfigurationsschritte im Detail beschreiben.

\subsection{Allgemeine Konfiguration}

\subsection{Kontakte Anlegen}

\subsection{Empfangsgruppen Zusammenstellen}

\subsection{Kanäle anlegen}

\subsection{Zonen zusammenstellen}

\subsection{Scanlisten zusammenstellen}

\appendix
\printindex

\end{document}