\section{Roaming} \label{sec:roaming}
Viele Relais in einer Region haben die selben Sprechgruppen abonniert, damit eine repeatertransparente Nutzung dieser Sprechgruppen möglich ist. Es ist also egal welchen Repeater Sie auf Ihrem Funkgerät ausgewählt haben, Sie können immer die gleichen Sprechgruppen verwenden. In der Region Berlin \& Brandenburg wäre dies die Sprechgruppe TG2621. 

Es wäre also sinnvoll eine Liste zu erstellen, in der alle Repeater eingetragen werden, die eine bestimmte Sprechgruppe abonniert haben. Wenn dann noch das Funkgerät automatisch einen erreichbaren Repeater aus dieser Liste auswählen könnte, dann könnte man mit dem Auto in dieser Region unterwegs sein und wäre immer automatisch mit dieser Sprechgruppe verbunden. Dieses Feature nennt sich \adef{Roaming} und wird von einigen meist etwas teureren Funkgeräten unterstützt (z.B., die AnyTone Geräte). Die günstigsten DMR Funkgeräte chinesischer Produktion unterstützen dieses Feature meist nicht. 

Um das Roaming nutzen zu können, werden zunächst alle Kanäle mit einer bestimmten Sprechgruppe in einer Liste zusammengefasst. Dies könnte eigentlich automatisch geschehen, aber die Konfigurationssoftware für diese Funkgeräte ist wirklich nicht sehr benutzerfreundlich. 

Wenn nun die Signalstärke eines bestimmten Repeaters dieser Liste unter einen Schwellwert (meist $-105dBm$) fällt, fängt das Funkgerät an, alle Kanäle der Roamingliste abzuklappern, bis es einen Repeater findet, dessen Signalstärke größer ist als der Schwellwert. Dies geschieht aber nur, wenn das Funkgerät auf \emph{stand-by} ist. Das heißt, wenn weder etwas auf dem aktuellen Kanal empfangen wird noch gesendet wird. 

Hat es einen stärkeren Repeater in der Liste gefunden, wechselt das Funkgerät automatisch den Kanal des neuen Repeater. Dies muss nicht unbedingt der stärkste Repeater der Liste am aktuellen Standort sein. Lediglich der Schwellwert ist entscheidend. Wird kein genügend starker Repeater gefunden, verbleibt das Funkgerät auf dem aktuellen Kanal. 

Dieses Roaming kann auch auf \emph{manuelles Roaming} eingestellt werden. Das heißt, das Roaming startet erst, wenn die Signalstärke des aktuellen Repeaters unter den Schwellwert sinkt und die PTT-Taste gedrückt wird oder ein manueller Suchlauf aus dem Menü heraus gestartet wird. 