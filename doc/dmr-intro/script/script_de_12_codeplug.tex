\section{Codeplug Programmierung} \label{sec:codeplug}
Nachdem Sie sich mit den Konzepten und dem technischen Hintergrund von DMR auseinandergesetzt haben, geht es nun an die Konfiguration Ihres Funkgerätes. Dies geschieht üblicherweise nicht über das Bedienfeld des Funkgerätes, sonder mit Hilfe einer separaten Software, der sogenannten \adef{CPS} oder \emph{codeplug programming software}. 

Doch bevor Sie loslegen können benötigen Sie wie alle DMR Teilnehmer eine eindeutige Nummer, die DMR ID.
\begin{hinweis}
 Ihre persönliche und eindeutige DMR ID erhalten Sie unter \url{https://register.ham-digital.org/}. Da Sie nachweisen müssen, dass Sie lizenzierter Funkamateur sind, müssen Sie bei der Anmeldung ihre eingescannte \emph{Zulassung zum Amateurfunkdienst} hochladen.
\end{hinweis}
Diese erhalten Sie in der Regel innerhalb von 24 Stunden per Mail. Sobald Sie eine DMR ID erhalten kann es los gehen.

Da dieses Script für Einsteiger gedacht ist, ist es wahrscheinlich das Sie kein top-shelf Motorola Gerät, sonder eher ein günstiges Gerät der einschlägig bekannten chinesischen Hersteller besitzen. 

\begin{achtung}
 Falls Sie noch kein DMR fähiges Funkgerät besitzen und mit dem Gedanken spielen eines zu kaufen, achten Sie unbedingt darauf, dass es DMR \textbf{Tier I \& II}\footnote{Wie so häufig ist DMR nicht ein Standard sondern eine ganze Familie von aufeinander aufbauenden Standards. DMR Tier I beschreibt im wesentlichen den DMR Simplexbetrieb und Tier II dann den Repeaterbetrieb mit zwei Zeitschlitzen. Sie benötigen also unbedingt Tier II für den Repeaterbetrieb.} unterstützt. Ignorieren Sie etwaiges Marketing-Bla-Bla der Hersteller und schauen Sie in den technischen Details nach ob dort DMR \textbf{Tier I \& II} erwähnt wird. Falls nicht oder nicht eindeutig, lassen Sie die Finger von diesem Gerät! Dies gilt vor allem für das Baofeng BR-5R aber nicht für das Baofeng/Radioddity RD5-R\footnote{Manchmal sind es die kleinen Unterschiede die entscheidend sind.}. 
\end{achtung}

Der Hersteller Ihres Gerätes wird auf seiner Webseite die Software, die Sie zur Konfiguration benötigen, zum Download bereitstellen. Diese Software wird \emph{CPS} oder \emph{codeplug programming software} genannt. Gegebenenfalls finden Sie dort auch Firmwareupdates für Ihr Gerät. Viele Hersteller bieten für jedes einzelne Modell eine separate CPS an oder gar für jede Variation eines Modells. Achten Sie also genau darauf welche CPS Sie herunterladen. Die Konfiguration dieser Geräte unterscheidet sich von Gerät zu Gerät und mehr noch von Hersteller zu Hersteller. Jedoch sind die wesentlichen Einstellungen für Geräte dieser Klasse sehr ähnlich.

Wenn Sie die CPS zum ersten mal starten, werden Sie wahrscheinlich zwei Dinge feststellen. Erstens, das Bedienkonzept dieser Software ist aus dem letzten Jahrtausend (Windows 3.11) und Zweitens, es gibt eine Unmenge an obskuren Optionen deren Funktion nicht ersichtlich ist und die größtenteils nicht Dokumentiert sind. Wenn Sie des Englischen nicht mächtig sind, werden Sie auch eine deutsche Übersetzung des Programms vermissen. Aber keine sorge, die englische Übersetzung ist meist auch so schlecht, dass es keinen Unterschied macht ob sie Englisch lesen können oder nicht.

Die Konfiguration Ihres Funkgerätes erfolgt in 5-6 Schritten:
\begin{enumerate}
 \item Allgemeine Einstellungen,
 \item Kontakte anlegen,
 \item Empfangsgruppen festlegen,
 \item alle Kanäle anlegen,
 \item Kanäle in Zonen einteilen und
 \item optional Scanlisten anlegen.
\end{enumerate}

In den folgenden Abschnitten möchte ich die einzelnen Konfigurationsschritte im Detail beschreiben.

\subsection{Allgemeine Konfiguration}

\subsection{Kontakte Anlegen}

\subsection{Empfangsgruppen Zusammenstellen}

\subsection{Kanäle anlegen}

\subsection{Zonen zusammenstellen}

\subsection{Scanlisten zusammenstellen}